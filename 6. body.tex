\titleformat{\chapter}[block]{\bfseries\fontsize{16pt}{16pt}\selectfont\center}{Chapter\ \thechapter{}.}{1em}{\setstretch{2.0}\bfseries\fontsize{16pt}{16pt}\selectfont\bfseries}[\fontsize{11pt}{11pt}\selectfont]
\titleformat{\section}[block]{\bfseries\fontsize{14pt}{14pt}\selectfont}{\thesection{}.}{1em}{\setstretch{2.0}\bfseries\fontsize{14pt}{14pt}\selectfont\bfseries}[]
\titleformat{\subsection}[block]{\fontsize{11pt}{11pt}\selectfont}{\thesubsection{}.}{1em}{\fontsize{11pt}{11pt}\selectfont}[]

\setstretch{2.0}
\fontsize{11pt}{11pt}\selectfont
\setlength{\parindent}{1.5em}
\setlength{\parskip}{0pt}
\chapter{INTRODUCTION}
\section{연구목적과 연구방법론}
\subsection{연구방법론}

여기부터 본문 작성하시기 바랍니다. 현재 서식은 본문 조판 서식입니다(본 본문 서식은 
신명조 11point 기준으로 줄간격 200\% 38자* 25행으로 되어있으며, 본 페이지에
적용되어 있으니 여기에 본문을 작성하시면 됩니다)\par
본문 및 각 문단의 시작은 3space를 띄우고 작성하시기 바랍니다.
각주 및 참고자료에 사용되는 기재방식은 학문분야별, 학회별로 기준이 상이하므로
각 학과의 기준에 따라 작성하시기 바랍니다.\par
본문제목은 한 줄을 띄우고 큰 제목(예: 제1장 서론)을 작성하고, 두 줄 아래에 
중간제목을 작성하고, 한 줄 아래에 소제목 또는 본문을 시작하시기 바랍니다.\par
The main body of the text starts here. The main text should be typed
with 신명조 11 point which is composed of line space 200\% 38characters*25lines.\par
This page is typeset with the above font sizes, so you can write the text by
using the typeset form of this page.\par
Leave two blank spaces at the beginning of each paragraph, and use a font size
of 11 points.\par
The conventions used for footnotes and references may vary in 
academic disciplines and societies. Thus, please follow the convention used in
your department.


\subsubsection{Subsubsection}
\chapter{Design}
\chapter{Experiments}
\begin{figure}
    \centering
    \caption{Figure 1}
    \label{fig:enter-label}
\end{figure}

\begin{figure}
    \centering
    \caption{Figure 2}
    \label{fig:enter-label2}
\end{figure}

\begin{table}[]
    \centering
    \begin{tabular}{c|c}
         &  \\
         & 
    \end{tabular}
    \caption{Table 1}
    \label{tab:my_label}
\end{table}

\begin{table}[]
    \centering
    \begin{tabular}{c|c}
         &  \\
         & 
    \end{tabular}
    \caption{Table 2}
    \label{tab:my_label2}
\end{table}