\documentclass{report}
% 여백은 년도별, 학과별 다를 수 있음 
\usepackage[a4paper, left=3cm, right=3cm, top=53mm, bottom=53mm]{geometry}
\usepackage{fix-cm}
\usepackage{graphicx, atbegshi} % Required for inserting images
\usepackage{fontspec}
\usepackage{kotex}
\usepackage[CJKspace=true]{xeCJK}
\usepackage{tocloft}
\usepackage{setspace}
\usepackage{xcolor}
\usepackage{titlesec}
\usepackage{indentfirst}


% HY신명조는 한글에서 사용하거나, 유료 다운로드 해야함
% 나눔 명조로 대체 
% 직접 유료로 HY신명조를 사용하는 사람은 overleaf 의 폴더에 해당 폰트 파일을 업로드 후
% \setmainfont{폰트파일이름.ttf} 처럼 그 폰트 파일로 수정 
\setmainfont{NanumMyeongjo}
\setCJKmainfont{NanumMyeongjo}

\begin{document}
% 저자 정보 아래에서 수정 
% 논문 제목
\newcommand{\thesistitle}{Dissertation Title}
% 학생 이름
\newcommand{\studentname}{Gildong Hong}
% 학과
\newcommand{\departmentname}{Department of Electrical and Computer Engineering}
% 지도교수님 성함 
\newcommand{\advisorname}{Gilsan Jang}
\newcommand{\committeechair}{King Kong}
\newcommand{\committeememberone}{Queen Kong}
\newcommand{\committeemembertwo}{Thunder bolt}
\newcommand{\committeememberthree}{Elec tron}

\pagenumbering{gobble}
% 외표지
% 고정 제목 
\begin{center}
\vbox to 15.7mm {
\vfill
\fontsize{16pt}{16pt}\selectfont Ph.D. Dissertation
}
\nointerlineskip

\vspace{15.5mm}

% 실제 논문 제목 
\vbox to 70.1mm {
\vspace{0.5mm}
\vspace{8.4pt}
\fontsize{22pt}{22pt}\selectfont\thesistitle{}
\vfill
}
\nointerlineskip

% 이름 
\vbox to 53.5mm {
\vspace{0.5mm}
\vspace{7.2pt}
\fontsize{16pt}{16pt}\selectfont\studentname{}
\vfill
}
\nointerlineskip

% 학과 
\vspace{0.5mm}
\setlength{\parskip}{7.2pt}
\setstretch{1.6}
\fontsize{16pt}{16pt}\selectfont
\departmentname{}\par
The Graduate School\par
Sungkyunkwan University\par
\end{center}


% 내표지
% 고정 제목 
\begin{center}
\vbox to 15.7mm {
\vfill
\fontsize{16pt}{16pt}\selectfont Ph.D. Dissertation
}
\nointerlineskip

\vspace{15.5mm}

% 실제 논문 제목, 바로 아랫줄의 Dissertation Title 을 수정
\vbox to 70.1mm {
\vspace{0.5mm}
\vspace{8.4pt}
\fontsize{22pt}{22pt}\selectfont\thesistitle{}
\vfill
}
\nointerlineskip

% 이름 
\vbox to 53.5mm {
\vspace{0.5mm}
\vspace{7.2pt}
\fontsize{16pt}{16pt}\selectfont\studentname{}
\vfill
}
\nointerlineskip

% 학과 
\vspace{0.5mm}
\setlength{\parskip}{7.2pt}
\setstretch{1.6}
\fontsize{16pt}{16pt}\selectfont
\departmentname{}\par
The Graduate School\par
Sungkyunkwan University\par
\end{center}

% 심사청구서
\setlength{\parskip}{0pt}
\vspace{13.7mm}
\vspace{8.4mm}

\begin{center}
\vbox to 31.8mm {
    \vspace{0.5mm}
    \setlength{\parskip}{7.2pt}
    \fontsize{22pt}{22pt}\selectfont
    \thesistitle
    \vfill
}
\nointerlineskip

\vbox to 22.5mm {
    \setstretch{1.6}
    \fontsize{16pt}{16pt}\selectfont
    \vfill
    \studentname
    \vfill
}
\nointerlineskip

\vbox to 81.2mm {
    \setstretch{1.6}
    \fontsize{14pt}{14pt}\selectfont
    \setlength{\parskip}{8.4pt}
    \vfill
    A Ph.D. Dissertation Submitted to the \departmentname
    and the Graduate School of Sungkyunkwan University in
    partial fulfillment of the requirements for the degree
    of Ph.D. in Engineering\\ 
    \hfill\break
    October 2023
    \vfill
}
\nointerlineskip
\vbox to 32.3mm {
    \setstretch{1.6}
    \fontsize{16pt}{16pt}\selectfont
    \setlength{\parskip}{7.2pt}
    \vfill
    Supervised by\\
    \advisorname\\
    Major Advisor
    \vfill
}
\nointerlineskip


\end{center}


% 심사위원 인증서
% 심사위원 추가 커맨드 
\newcommand{\judgesign}[1]{
    \vbox to 9.7mm {
        \setstretch{1.5}
        \setlength{\parskip}{3.6pt}
        \fontsize{12pt}{12pt}\selectfont
        \vspace{0.5mm}
        \vfill
        \hspace{60.5mm}
        \hfill{}\textcolor{lightgray}{#1 signature}\hfill{}\hfill{}
        \hspace{7.8mm}
        \vfill
        \vspace{0.5mm}
    }
}

\newcommand{\judgename}[2]{
    \vbox to 9.7mm {
        \setstretch{1.5}
        \setlength{\parskip}{3.6pt}
        \fontsize{12pt}{12pt}\selectfont
        \vspace{0.5mm}
        \vfill
        \hspace{52.5mm}
            #1: #2
        \vfill
        \vspace{0.5mm}
    }
}

\newcommand{\judgeitem}[3]{
    \judgesign{#1}
    \nointerlineskip
    \vbox to 0.5mm {
        \hspace{60.5mm} \hrulefill \hspace{7.8mm}
        \vfill
    }
    \nointerlineskip
    \judgename{#2}{#3}
}

% 포맷 시작 
\vspace{5.9mm}
\vspace{4.5mm}
\vbox to 24.4mm {
\setstretch{1.5}
\setlength{\parskip}{3.6pt}
\fontsize{16pt}{16pt}\selectfont
\vspace{0.5mm}
\begin{center}
    This certifies that the Ph.D. Dissertation\\
    of \studentname{} is approved
\end{center}

\vfill
}
\nointerlineskip
\vspace{9.7mm}
\judgeitem{심사위원장}{Committee Chair}{\committeechair}
\vspace{4.7mm}
\nointerlineskip
\judgeitem{심사위원 1}{Committee Member 1}{\committeememberone}
\nointerlineskip
\vspace{4.7mm}
\judgeitem{심사위원 2}{Committee Member 2}{\committeemembertwo}
\nointerlineskip
\vspace{4.7mm}
\judgeitem{심사위원 3}{Committee Member 3}{\committeememberthree}
\nointerlineskip
\vspace{4.7mm}
\judgeitem{지도교수}{Major Advisor}{\advisorname}
\nointerlineskip

\setstretch{1.5}
\fontsize{16pt}{16pt}\selectfont
\setlength{\parskip}{0pt}
\hfill\break
\begin{center}
    The Graduate School\\
    Sungkyunkwan University\\
    December 2023
\end{center}

\pagenumbering{roman}
% 목차 (table of contents)
% 기본 레이아웃 설정 
\newcommand{\listspacing}{1mm}
\newcommand{\listdefaultfontsize}{11pt}
\newcommand{\chapfontsize}{\fontsize{\listdefaultfontsize}{\listdefaultfontsize}\selectfont\bfseries}
\newcommand{\secfontsize}{\fontsize{\listdefaultfontsize}{\listdefaultfontsize}\selectfont}
\newcommand{\subsecfontsize}{\fontsize{\listdefaultfontsize}{\listdefaultfontsize}\selectfont}
\newcommand{\figfontsize}{\fontsize{\listdefaultfontsize}{\listdefaultfontsize}\selectfont}
\newcommand{\tabfontsize}{\fontsize{\listdefaultfontsize}{\listdefaultfontsize}\selectfont}

\renewcommand{\cftchappresnum}{Chapter }
\renewcommand{\cftchapaftersnum}{.}
\renewcommand{\cftchapnumwidth}{5em}
\renewcommand{\cftbeforechapskip}{\listspacing}
\renewcommand{\cftchapfont}{\chapfontsize}

\renewcommand{\cftbeforesecskip}{\listspacing}
\renewcommand{\cftsecfont}{\secfontsize}

\renewcommand{\cftbeforesubsecskip}{\listspacing}
\renewcommand{\cftsubsecfont}{\subsecfontsize}

\renewcommand{\cftbeforefigskip}{\listspacing}
\renewcommand{\cftfigfont}{\figfontsize}

\renewcommand{\cftbeforetabskip}{\listspacing}
\renewcommand{\cfttabfont}{\tabfontsize}

% 연도, 학과에 맞게 수정

% 목차 제목 , 가운데 정렬
\setlength{\cftaftertoctitleskip}{16pt}
\renewcommand{\cfttoctitlefont}{\fontsize{16pt}{16pt}\selectfont\bfseries}
\renewcommand{\contentsname}{\hfill{}Table of Contents\hfill}
\renewcommand{\cftaftertoctitle}{\hfill\newline}


% 목차 생성 (toc)
\tableofcontents

% 새로운 페이지로 이동
\clearpage

% 그림 목차 , 가운데 정렬
\renewcommand{\cftloftitlefont}{\fontsize{16pt}{16pt}\selectfont\bfseries}
\renewcommand{\listfigurename}{\hfill List of Figures \hfill}
\listoffigures

% 새로운 페이지로 이동
\clearpage

% 표 목차 , 가운데 정렬
\renewcommand{\cftlottitlefont}{\fontsize{16pt}{16pt}\selectfont\bfseries}
\renewcommand{\listtablename}{\hfill List of  Tables \hfill}
\listoftables

\pagenumbering{arabic}
% 영문 초록 (abstract)
% 좌상단 abstract 
\setstretch{1.6}
\fontsize{14pt}{14pt}\selectfont{}\textbf{Abstract}
\setstretch{1.0}

% 논문 제목
\setlength{\parskip}{11.5pt}
\fontsize{11pt}{11pt}\selectfont{}\textbf{ }
\setlength{\parskip}{0pt}
\begin{center}
\fontsize{16pt}{16pt}\selectfont{}\textbf{\thesistitle}
\end{center}

% 초록 본문 
\setlength{\parskip}{0pt}
\setstretch{2.0}
\setlength\parindent{1em}
\fontsize{11pt}{11pt}\selectfont{}
여기부터 논문요약을 2page 이내(주제어 기재란 포함)로 작성하시기 바랍니다.\par
논문요약의 끝에는 논문의 핵심이 되는 단어(주제어) 2개이상 5개이하를 쉼표로
구분하여 한글 또는 영어로 작성하시기 바랍니다. 이 주제어는 논문 제목과 함께 본
논문을 검색할 수 있는 검색용 단어로 사용됩니다.\par
Leave two blank spaces at the beginning of each paragraph, and
use the font size of 11 points.\par
The length of the abstract should not be longer than 2 pages.\par
List 2-5 keywords separated by commas.

% 최대 5개의 키워드
\fontsize{11pt}{11pt}\selectfont{}
\textbf{keywords: 키워드1, 키워드2, 키워드3, 키워드4, 키워드 5}

% 본문 
\titleformat{\chapter}[block]{\bfseries\fontsize{16pt}{16pt}\selectfont\center}{Chapter\ \thechapter{}.}{1em}{\setstretch{2.0}\bfseries\fontsize{16pt}{16pt}\selectfont\bfseries}[\fontsize{11pt}{11pt}\selectfont]
\titleformat{\section}[block]{\bfseries\fontsize{14pt}{14pt}\selectfont}{\thesection{}.}{1em}{\setstretch{2.0}\bfseries\fontsize{14pt}{14pt}\selectfont\bfseries}[]
\titleformat{\subsection}[block]{\fontsize{11pt}{11pt}\selectfont}{\thesubsection{}.}{1em}{\fontsize{11pt}{11pt}\selectfont}[]

\setstretch{2.0}
\fontsize{11pt}{11pt}\selectfont
\setlength{\parindent}{1.5em}
\setlength{\parskip}{0pt}
\chapter{INTRODUCTION}
\section{연구목적과 연구방법론}
\subsection{연구방법론}

여기부터 본문 작성하시기 바랍니다. 현재 서식은 본문 조판 서식입니다(본 본문 서식은 
신명조 11point 기준으로 줄간격 200\% 38자* 25행으로 되어있으며, 본 페이지에
적용되어 있으니 여기에 본문을 작성하시면 됩니다)\par
본문 및 각 문단의 시작은 3space를 띄우고 작성하시기 바랍니다.
각주 및 참고자료에 사용되는 기재방식은 학문분야별, 학회별로 기준이 상이하므로
각 학과의 기준에 따라 작성하시기 바랍니다.\par
본문제목은 한 줄을 띄우고 큰 제목(예: 제1장 서론)을 작성하고, 두 줄 아래에 
중간제목을 작성하고, 한 줄 아래에 소제목 또는 본문을 시작하시기 바랍니다.\par
The main body of the text starts here. The main text should be typed
with 신명조 11 point which is composed of line space 200\% 38characters*25lines.\par
This page is typeset with the above font sizes, so you can write the text by
using the typeset form of this page.\par
Leave two blank spaces at the beginning of each paragraph, and use a font size
of 11 points.\par
The conventions used for footnotes and references may vary in 
academic disciplines and societies. Thus, please follow the convention used in
your department.


\subsubsection{Subsubsection}
\chapter{Design}
\chapter{Experiments}
\begin{figure}
    \centering
    \caption{Figure 1}
    \label{fig:enter-label}
\end{figure}

\begin{figure}
    \centering
    \caption{Figure 2}
    \label{fig:enter-label2}
\end{figure}

\begin{table}[]
    \centering
    \begin{tabular}{c|c}
         &  \\
         & 
    \end{tabular}
    \caption{Table 1}
    \label{tab:my_label}
\end{table}

\begin{table}[]
    \centering
    \begin{tabular}{c|c}
         &  \\
         & 
    \end{tabular}
    \caption{Table 2}
    \label{tab:my_label2}
\end{table}

% 국문 초록 (abstract)
\begin{abstract}
    
\end{abstract}





\end{document}
